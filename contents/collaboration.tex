%!TEX root = ../main.tex

{\Large
\begin{center}
\textbf{OTHER SUPPLEMENTARY DOCS}
\end{center}
}

\noindent---------------------------------------------------------------------------------------------------------------------------------

\section*{List of Project Personnel}

\begin{enumerate}
   \item Endadul Hoque; Syracuse University; PI
   \item Omar Chowdhury; University of Iowa; Co-PI
\end{enumerate}

\newpage

\section*{Collaboration Plan}

\hoque{Re-write for us}
The principal investigators are located at the University of Florida
(UF), North Carolina State University (NC State), Arizona State University
(ASU), Purdue University and the University of Iowa. This grant is
designed to bring each of the involved groups into tighter, more
meaningful collaboration. For instance, the UF and NC State PIs have a long
history of collaboration, and have together co-authored many of the most
cited papers in cellular and mobile security. 

\begin{figure}
    \centering
    %\input{fig/schedule}
    \includegraphics[width=\textwidth]{fig/collaboration}
    \caption{Collaboration of PIs on Tasks}
    \label{fig:collab}
\end{figure}

Figure~\ref{fig:collab} details the mapping of planned collaborations to
Tasks.  While distinct, the thrusts and their tasks have
inter-dependencies that will engage the strengths of the different PIs.
Thrust~\ref{th:systems} will be led by PIs Enck, Butler, and Doup\'e
based on their extensive experience performing systems verification in
related domains.  Thrust~\ref{th:protocols} will be led by PIs Bertino
and Chowdhury based on their expertise in applying formal verification
techniques in cellular protocol and other network protocols.
Thrust~\ref{th:identity} will be led by PIs Doup\'e, Reaves, and Traynor
based on their expertise in security of cellular systems.  However,
these activities will not operate in isolation.  Both verification
thrusts (\ref{th:systems} and~\ref{th:protocols}) will frequently
require the deep cellular expertise of PIs Reaves and Traynor.
Furthermore, Thrusts~\ref{th:systems} and~\ref{th:protocols} will
frequently require coordination among PIs across the systems and
protocols boundaries.  For example, we seek to use systems verification
to support trust assumption in protocols, and protocol verification to
support trust assumptions in systems. Finally, the
Thrust~\ref{th:identity} tasks will frequently engage the PIs focusing
on the verification thrusts, leveraging the results of their individual
analyses.


The full team plans to collaborate through a variety of mechanisms on
this project. First, members and their funded students will be involved
in weekly task meetings, regular progress reports, presentations and
demonstrations. Teams will use Slack and Skype to ensure smooth
interaction across institutions. We note that we have been using these
tools in preparation for the submission of this proposal, and all team
members regularly participate through these channels. 

Second, each of the task leads (listed above) will meet with PI
Traynor monthly to discuss overall progress. These meetings will be used
to generate reports for the overall project, which will be PI Traynor's
responsibility. It is the expectation of the PIs that the majority of
publications resulting from this work will be co-authored with other PIs on
the proposal.

The team will also ensure collaboration through the following additional
events:

\vspace{0.2cm} 
\noindent \textbf{Annual research retreat:} We will organize an annual
two-week long summer research retreat for all members of the project.
Our goal is to ensure that all team members, especially students, have
the opportunity to meaningfully work together. The first such meeting is planned
at the University of Florida in Summer of 2020, with subsequent
meetings rotating to the campuses of the other participants. A public
report will be generated from this meeting, and we intend to also invite
external speakers. We have explicitly created a budget line item (see
the University of Florida's budget) to ensure that sufficient funds are
set aside annually for this event. 

\vspace{0.2cm} \noindent \textbf{Quarterly virtual meetings and
reports:} We plan to have quarterly detailed reports and presentations
from all team members. These reports will be hosted in our internal
discussion forums to share project progress with all personnel. Mature
reports will be made available to the public.

\vspace{0.2cm} \noindent \textbf{Broad collaboration with partners:} In
addition to our broad dissemination of our findings through traditional
academic venues, our results will be shared regularly with our partners
across governmental, NGO, and corporate sectors. Through our combined
efforts, it is our hope that the techniques and tools developed in this
work are broadly adopted within mobile payments (and beyond).

We note that the above collaboration will be assisted by the Program
Coordinator, whose affiliated costs have been included in the University
of Florida's budget.


%\newpage
%
%\section*{Topic Areas}
%
%\noindent {\bf Primary:} Networked Systems; Security; IoT Systems; Runtime Monitoring

%\noindent {\bf Secondary:} Authentication and Biometrics

%\newpage

%\begin{comment}
%\section*{Broadening Participation in Computing (BPC) Plan}
%
%The PI Team and the Deparment of Computer and Information Science and
%Engineering (CISE) at the University of Florida take the mission of
%broadening participation in computing seriously. In particular, both the
%PI team and the Department have and continue to focus on outreach to
%both women and African American students.
%
%\noindent {\bf PI Team:} PI Traynor currently advises two female African
%American Ph.D.\ students and one Hispanic male. PI Traynor will continue
%to expand representation by using this grant to fund an incoming female
%US citizen Ph.D. student (Cassidy Gibson - Fall 2019) and will serve as
%research for credit for an undergraduate female student (Jessica
%O'Dell).  PI Shrimpton has previously mentored four female students
%---~one undergraduate honors thesis, two MS theses, one Ph.D.\
%dissertation~--- and currently serves on the Ph.D.\ committee of a
%female, African American Ph.D.\ student.  In his three years at the
%University of Florida, he has vigorously recruited multiple rising-PhD
%students from underrepresented groups (they ultimately accepted offers
%from other top-tier schools); he aims for this grant to support one such
%student. PI Bindschaedler, who is in his first year as an Assistant
%Professor, will be recruiting heavily as he establishes his research
%program. One of his goal will be to recruit Florida students from
%overlooked and underrepresented groups (for example Hispanics and
%Latinos constitute about 23\% of Florida's population). He intends to
%meet and recruit such students by giving talks at universities in
%South Florida.  PI Bindschaedler has successfully worked on research with
%undergraduates in the past and he will intensely continue efforts to
%mentor undergraduates and promote undergraduate research in computer
%security. All PIs will track the total number of students from graduate
%and undergraduate populations engaged with during the lifetime of this
%grant.
%
%All PIs will continue efforts to broaden participation in computer
%science as an area of study.   One recent example of this is our 2019
%Workshop on Password Security for Middle School girls in Gainesville,
%FL.~\footnote{https://twitter.com/JasmineDBowers/status/1113900326195744768}
%The workshop will be repeated in May of 2019 for a different group of 14
%middle school women, all of whom rated the workshops as either
%``valuable'' or ``extremely valuable'' in their exit surveys.  In line
%with the topics in this proposal, a future workshop (in Spring 2020)
%will focus on voice security and transforming voice.  We will also use a
%portion of the workshop to focus on the broader picture of what a degree
%in Computer Science can enable. We will continue this workshop program
%during the proposed period of this work, and as metrics will include the
%attendance and exit survey results in our NSF annual report.
%
%All PI Team members actively work with the Herbert Wertheim College of
%Engineering's recruiting director in order to cultivate diversity. Mike
%Nazareth, Director of Graduate Recruiting and Undergraduate Research,
%has developed an innovative program in which contact information for
%high-performing students from around the country are exchanged and
%collected, allowing us to directly identify, contact, and encourage
%underrepresented groups to apply for graduate school in CISE. This
%program, now in its second year, allows the PIs to not only speak with
%specific underrepresented groups, but also to open dialog with
%high-performing students at many colleges and universities traditionally
%overlooked in the discipline. As an example, PI Traynor's incoming
%female student was recruited through this program from Suwanee: The
%University of the South. We will continue to actively participate in
%this larger diversity initiative throughout and beyond the lifetime of
%this proposed work, and will track interaction with the number of
%students from underrepresented universities.
%
%
%\noindent {\bf CISE Department:} The CISE Department at the University
%of Florida has and continues to take concrete efforts to broaden
%participation in computing. Beginning in 2014, the Department adopted
%the Institute for African-American Mentoring in Computing Sciences's
%(iAAMCS) Guidelines for Successfully Mentoring Black/African-American
%Computing Sciences PhD Students. These include strategic recruitment,
%establishing a community, and providing holistic advising. This, in
%concert with establishing marketing materials and participating in the
%DREU program show a continued commitment to increasing diversity in
%computing. 
%
%These efforts have sparked significant change.  CISE has more African
%American Ph.D. students in computer science than any other university in
%the United States (representing approximately 14\% of Ph.D. students).
%Moreover, approximately 32\% of Ph.D. students in CISE are women,
%placing the department at 1.5 times the national average. These
%statistics are advertised in one of our annual mailers, and are a great
%point of pride for the Department.
%
%Both the PI Team and the Department will continue to expand
%participation through these means. Specifically, the PIs intend to
%recruit students via future DREU opportunities.
%
%\end{comment}
%
